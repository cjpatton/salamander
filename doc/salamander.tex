\documentclass[letter]{article}
\usepackage[utf8]{inputenc}
\usepackage{tikz}
\usepackage{ulem}

\usepackage{hyperref}
\hypersetup{
    colorlinks,%
    citecolor=black,%
    filecolor=black,%
    linkcolor=black,%
    urlcolor=black
}

\def\dashuline{\bgroup 
  \ifdim\ULdepth=\maxdimen  % Set depth based on font, if not set already
	  \settodepth\ULdepth{(j}\advance\ULdepth.4pt\fi
  \markoverwith{\kern.15em
	\vtop{\kern\ULdepth \hrule width .3em}%
	\kern.15em}\ULon}

\newcounter{foot}
\setcounter{foot}{1}

\author{Christopher Patton}
\date{\today}
\title{Salamander: computer vision for wildlife research}
	
\begin{document}
\maketitle

\begin{abstract}
This document outlines an image prcessing application called Salamander. Salamander
is comprised of a set of algorithms for automatically detecting and tracking 
targets of interest in a video stream. It is particularly useful for filtering 
and segmenting video produced by field cameras for wildlife and behaviorial 
research. 
\end{abstract}

\tableofcontents

\section{Introduction}
stuff

\section{Algorithms}
stuff

\subsection{Image processing}
delta(im1, im2) - binthresh(im) - binmorph(im)

\subsection{Detection}
connectedComponents(im). How to handle noise. 

\subsection{Tracking}
Shifting over merged blobs. 
Multiple targets. 

\subsection{Segmentation}
Data structures. 
Testing if gap has target. 
 
\section{The Salamander toolkit}
stuff

\subsection{Build and install} 
stuff

\subsection{Usage} 
stuff

\subsubsection{\texttt{binthresh} and \texttt{binmorph}}
stuff

\subsubsection{\texttt{filter}}
stuff

\subsubsection{\texttt{detect}}
stuff

\subsubsection{\texttt{segment}}
stuff

\subsection{Processing a feed in realtime}
stuff

\section{Machine learning}
I still need to research how and if to use this. 

\end{document}
